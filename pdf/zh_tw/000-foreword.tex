\chapter{寫在前面的話} \label{fore}


\begin{kaitext}

本書起源于中國電子工業出版社出版的一本書:《自己動手寫操作系統》(于淵著)。我對《自己動手寫操作系統》這本書中使用商業軟件做為演示平台比較驚訝,因為不是每個人都買得起正版軟件的,尤其是窮學生。我想《自》所面向的主要受眾也應該是學生,那麼一本介紹只有商業軟件才能實現的編程技巧的書將會逼著窮學生去使用盜版,這是非常罪惡的行為~\frownie。

由于本人是一個~Linux~用戶,一個開源軟件的擁護者,所以就試著使用開源軟件實現這本書中的所有~demo~,並在\href{http://blog.solrex.cn}{自己的博客}上進行推廣。後來我覺得,為什麼我不能自己寫本書呢?這樣我就能插入漂亮的插圖,寫更詳盡的介紹而不怕篇幅過長,更容易讓讀者接受也更容易傳播,所以我就開始寫這本《~\BookName~》。

定下寫一本書的目標畢竟不像寫一篇博客,我將盡量詳盡的介紹我使用的方法和過程,以圖能讓不同技術背景的讀者都能通暢地完成閱讀。但是自己寫並且排版一本書不是很輕鬆的事情,需要耗費大量時間,所以我只能抽空一點一點的將這本書堆砌起來,這也是您之所以在本書封面看到本書版本號的原因~\smiley。

本書的最終目標是成為一本大學“計算機操作系統”課程的參考工具書,為學生提供一個~step by step~的引導去實現一個操作系統。這不是一個容易實現的目標,因為我本人現在並不自信有那個實力了解操作系統的方方面面。但是我想,立志百裡行九十總好過于躑躅不前。

《自己動手寫操作系統》一書開了個好頭,所以在前面部分,我將主要討論使用開源軟件實現《自》的~demo~。如果您有《自》這本書,參考閱讀效果會更好,不過我將盡我所能在本書中給出清楚的講解,盡量使您免于去參考《自》一書。

出于開放性和易編輯性考慮,本書採用~\LaTeX~排版,在成書前期由于專注于版面,代碼比較雜亂,可讀性不強,暫不開放本書~\TeX~源代碼下載。但您可以通過~SVN check out~所有本書相關的源代碼和圖片,具體方法請參見電子書主頁。

如果您在閱讀過程中有什麼問題,發現書中的錯誤,或者好的建議,歡迎您使用我留下的聯系方式與我聯系,本人將非常感謝。
\vskip 1cm
\noindent
\makebox[\textwidth][r]{楊文博}
\makebox[\textwidth][r]{個人主頁:\url{http://solrex.cn}}
\makebox[\textwidth][r]{個人博客:\url{http://blog.solrex.cn}}
\makebox[\textwidth][r]{2008~年~1~月~9~日}
\end{kaitext}

\begin{lined}{\textwidth}
\textbf{更新歷史}
\small
\begin{description}
    \item[Rev. 1]~確定書本排版樣式,添加第一章,第二章。
    \item[Rev. 2]~添加第三章保護模式。
\end{description}
\vspace{2ex}
\end{lined}

